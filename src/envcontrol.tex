%!TEX root = main.tex
% This file contains everything to do with controlling environments
% except from outsourcing stuff with \At and \Trigger:
% Namely deciding whether or not a problem, intro and/or solution should
% displayed

\makeatletter
% \At\EndProblem{\tighten@paragraph@solutions\tighten@paragraph}
\At\EndProblem{%
\if\exbank@opt@tightenparagraphs\@isTrue\relax%
  \if\@SolutionsOnly\@isFalse%
    \if\@DisplaySolutions\@isTrue%
      \penalty-300%
    \fi%
  \fi%
\fi%
}
\@ifundefined{figure}{}{
\edef\figure{\unexpanded{\tighten@paragraph@always}\unexpanded\expandafter{\figure}}
}
\@ifundefined{endfigure}{}{
\g@addto@macro\endfigure{\tighten@paragraph@always}
}
\gdef\isFalse{0}
\gdef\isTrue{1}
\gdef\DisplayProblem{\isTrue}
\gdef\@displayMetaCounter{\isFalse}

%%%% NB: Difference between \DisplaySolution and \DisplaySolutions
%%NB: DOC for \DisplaySolutions are moved down
\AtBeginDocument{
  \if\@DisplaySolutions\isFalse
    \@latex@warning{Hiding solutions. Show them with \string\DisplaySolutions}
  \fi
}
\global\let\do@ProcessCutFile = \ProcessCutFile
%% Problems
%:$triggers.=\Trigger\BeginPartproblem:\\ Triggers before a partproblem is inserted\\
%:$triggers.=\Trigger\VeryBeginPartproblem:\\ Triggers right after \BeginPartproblem. This is so that the user can do stuff before the actual headers start. The partproblem headers are invoked by \At\VeryBeginPartproblem\\
\newif\ifexb@isdisplaying
\newif\ifexb@hidenextsol
\global\exb@hidenextsolfalse
%
\gdef\exb@ifshowproblem#1#2{\exb@isdisplayingfalse\ifnum\pdfstrcmp{\DisplayProblem}{\@isTrue}=\z@\ifexb@opthides\else\exb@isdisplayingtrue\fi\fi%
\ifexb@isdisplaying#1\else#2\fi}
\gdef\showhideproblem#1{%
\exb@ifshowproblem{\let\ProcessCutFile\do@ProcessCutFile#1\global\exb@hidenextsolfalse}{\def\ProcessCutFile{}\global\exb@hidenextsoltrue}%
}
% \gdef\showhideproblem@old#1{%
% \ifnum\pdfstrcmp{\DisplayProblem}{\@isFalse}=\z@\def\ProcessCutFile{}\else%
%   \ifexb@opthides%
%     \def\ProcessCutFile{}%
%   \else%
%     \if\DisplayProblem\isFalse%
%         \def\ProcessCutFile{}%
%     \else%
%       \if\@SolutionsOnly\@isTrue\relax%
%         \def\ProcessCutFile{}%
%       \fi%
%         #1%
%     \fi
%   \fi\fi%
% }

% \At\BeginProblem{\if\@SolutionsOnly\@isFalse\relax\tighten@paragraph\fi}

%:§env
%:=\\begin{problem}
%:Inside the \keyRef{exercise directory}, you keep your exercises. Inside the exercise file you'd use a problem environment to write your partproblems. It might be a little confusing that you're using \begin{problem} instead of \begin{partproblem} when you're writing a partproblem, but it's less typing.
%\!begin{marker}\end{problem} has to be on it's own line without any leading spaces!\!end{marker}
% :-
\generalcomment{problem}{
\makeatletter
  %%%% NB: Difference between \DisplaySolution and \DisplaySolutions
  \edef\DisplaySolution{\@DisplaySolutions}
  \stepcounter{metacounter}
  %% BeginProblemHard is before regardless whether
  %% the problem is included or not.
  \Trigger\DecideProblemDisplay
  % \begingroup
     \showhideproblem{%
        % If the user haven't turned off part problems
        \if\exbank@opt@partProblems\isTrue
            \Trigger\BeginPartproblem
            \Trigger\VeryBeginPartproblem
        \else
            \Trigger\BeginProblem
            \Trigger\VeryBeginProblem
        \fi
      }
  }{
    \if\DisplayProblem\@isTrue\relax
      \if\exbank@opt@partProblems\@isTrue\relax%\ifexb@opthides\else
        \Trigger\EndPartproblem
        \tighten@paragraph\fi
      \else
        \Trigger\EndProblem
    \fi
}
\g@addto@macro\AfterproblemComment{\nextproblem{default}}
%:§env
%:=\\begin{solution}
%: Things inside here is only visible if \refCom{DisplaySolutions} are given before \begin{document}
%: \!begin{marker}\end{solution} has to be on it's own line without any leading spaces!\!end{marker}
%:-
%% Solutions
\generalcomment{solution}
{
\Trigger\AtBeginSolutionHard
\begingroup
  \if\@DisplaySolutions\isTrue
    \if\DisplayProblem\isFalse
      \xdef\DisplaySolution{\isFalse}
    \fi
  \fi
  \if\DisplaySolution\isTrue
    \ifexb@hidenextsol\else\Trigger\BeginSolution\fi
  \else
    \def\ProcessCutFile{}
  \fi
}{%\exb@ifshowproblem{
% \input{\CommentCutFile}
\if\DisplaySolution\isTrue
\Trigger\EndSolution
\tighten@paragraph@solutions
\vspace*{0.5em}
\fi
\endgroup

\Trigger\EndSolutionHard%
% \egroup
}
%% Problem introductions
%:§env
%:=\\begin{intro}
%:Sometimes you'd want to introcude your exercises and tell a little bit about it. Maybe have a figure there also. Those things should go inside this environment. This can be treated as a problem in terms of counting. See \refCom{makeset} for more info.
%\!begin{marker}\end{intro} has to be on it's own line without any leading spaces!\!end{marker}
%:-
\def\intromargin#1{%
\if\exbank@opt@doMargins\@isTrue%
\gdef\endintromargin{\endlist\endgroup}%
\begingroup\list{}{\leftmargin#1}\item[]%
\else%
\relax\global\let\endintromargin=\relax\fi%
}
\generalcomment{intro}{%
  %% Decide wether to treat intro as problem in
  %% include/exclude (except from \Trigger\BeginProblem ofc)
  \if\@countIntros\isTrue%
    \stepcounter{metacounter}%
    \Trigger\DecideProblemDisplay%
    % \begingroup%
    \gdef\exb@pre@intro{}%
    \gdef\exb@post@intro{}%
    \if\@displayMetaCounter\isTrue\relax%%
    \gdef\exb@pre@intro{{\leavevmode{\llap{{\Large\themetacounter}:\hspace*{-\pMarginLeft}}}}\ignorespaces}%
    \fi%
    \if\exb@showtags\isTrue%%
    % \exb@printCurrentTags defined in packageoptions.tex%
    \xdef\exb@post@intro{\exb@printCurrentTags}%
    \fi%
    \if\@spriteMode\@isTrue%
      \if\introarg\@isTrue%
        \stepcounter{partproblemcounter}%
      \fi%
    \fi%
    \gdef\@displayIntro{\@isFalse}
    \showhideproblem{%%
      \tighten@paragraph%%
      \Trigger\BeginIntro\exb@pre@intro\exb@post@intro%%
      \gdef\@displayIntro{\@isTrue}
    }%
  \exb@ifshowproblem{\intromargin{-1em}}{}
  \else%
  \intromargin{-1em}
  \fi%

}{
  \if\@countIntros\isTrue%
    \Trigger\EndIntro%
    \exb@ifshowproblem{\endintromargin}{}
  \else
    \endintromargin
  \fi%
  \ignorespaces%
% \newline
}
\g@addto@macro\AfterintroComment{\nextproblem{default}}
