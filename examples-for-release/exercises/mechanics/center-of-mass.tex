%!TEX root = ../../example.tex
\begin{intro}
	% This environment has an outdented left margin.
	In this exercise we will be looking at the center of mass of a thin rod of length \( \ell \) and total mass \( M \). The mass is distributed according to a mass per length \( \lambda(x) \). See figure \ref{fig:mechanics-rod} below
	% Note that the file in \input here is located in figures/mechanics/rod.tex and can be included as %!TEX root = ../../example.tex
% Very basic figure just to show principles
\usetikzlibrary{fit}
\makeatletter
\tikzset{%
  % when using \node[fitted node], one sets the bounding box of the node
  % and can use e.g. if this is set \node[fitted node](my-rect){}; then
  % my-rect.north east would be the top right corner of the bounding box
	fitted node/.style={
			inner sep=0pt,
			fit={(\pgf@pathminx,\pgf@pathminy)(\pgf@pathmaxx,\pgf@pathmaxy)},
		}
}
\makeatother
\newcommand\setnewlength[2]{\newlength#1\setlength#1{#2}}
\begin{tikzpicture}
  \newlength\lengthOffset
  \setlength\lengthOffset{-0.3cm}
  % Draw the rod
	\draw(0,0) rectangle (4cm,.5cm) node[fitted node](rod){};
  % Draw the line below showing the length
  \draw[|-|]([yshift=\lengthOffset]rod.south west) -- ([yshift=\lengthOffset]rod.south east) node[midway,below,fitted node](length line){};
  % Show "origo" and x=\ell
  \node[below] at (length line.south west){$x=0$};
  \node[below] at (length line.south east){$x=\ell$};
	\node at (rod.center) {$m$};
\end{tikzpicture}
 because _this_ file is in exercises/mechanics.
	% You can also include figures that are placed in the directory called figures/mechanics/center-of-mass/
	% This is due to \exercisebanksetup{figure root directory=figures} is set in the example.tex file
	\begin{figure}[h]
		\centering
		%!TEX root = ../../example.tex
% Very basic figure just to show principles
\usetikzlibrary{fit}
\makeatletter
\tikzset{%
  % when using \node[fitted node], one sets the bounding box of the node
  % and can use e.g. if this is set \node[fitted node](my-rect){}; then
  % my-rect.north east would be the top right corner of the bounding box
	fitted node/.style={
			inner sep=0pt,
			fit={(\pgf@pathminx,\pgf@pathminy)(\pgf@pathmaxx,\pgf@pathmaxy)},
		}
}
\makeatother
\newcommand\setnewlength[2]{\newlength#1\setlength#1{#2}}
\begin{tikzpicture}
  \newlength\lengthOffset
  \setlength\lengthOffset{-0.3cm}
  % Draw the rod
	\draw(0,0) rectangle (4cm,.5cm) node[fitted node](rod){};
  % Draw the line below showing the length
  \draw[|-|]([yshift=\lengthOffset]rod.south west) -- ([yshift=\lengthOffset]rod.south east) node[midway,below,fitted node](length line){};
  % Show "origo" and x=\ell
  \node[below] at (length line.south west){$x=0$};
  \node[below] at (length line.south east){$x=\ell$};
	\node at (rod.center) {$m$};
\end{tikzpicture}

		\caption{Rod with mass per length \label{fig:mechanics-rod}}
	\end{figure}
	This exercise consists of a total of \exercisepoints\ points.
\end{intro}
% Now we set points for this exercise
\nextproblem{points=1}
\begin{problem}
	Assume in this part problem that \( \lambda(x) \) is constant with regards to \( x \) (that is uniformly distributed mass).
	Use the definition of the center of mass to obtain the \( x \) coordinate for the center of mass.
\end{problem}
\begin{solution}
	\begin{align*}
		r_{\text{cm}} &=\frac1M\int_{0}^{M}x\dd m                               \\
		\lambda       &= \der mx = \frac M\ell                                  \\
		\intertext{Substituting $\dd m$ using $m=\frac M\ell x$ and $x(m)=\frac{m\ell}{M}$ gives}
		% \Rightarrow \der mx &= \frac M\ell x\\
		r_{\text{cm}} &=\frac{1}{M}\int_{x(0)}^{x(M)}x\frac M\ell\dd x          \\
		              &=\frac1\ell\int_{0}^{\ell}x\dd x=\frac1\ell\frac12\ell^2 \\
		              &=\uu{\frac12\ell}
	\end{align*}
\end{solution}
% Now we set points for this second exercise
\nextproblem{points=3}
\begin{problem}
	We now have a (still thin) rod that has linear mass distribution (\( \lambda(x)=x \)).
	Find the center of mass for this rod.
\end{problem}
\begin{solution}
	\begin{align*}
		\der mx           &=\lambda(x)=x                   \\
		\intertext{Using differential notation that mathematicians hate}
		\Rightarrow \dd m &=x\dd x                         \\
		\intertext{Again, substitution gives}
		r_{\text{cm}}     &=\frac1M\int_{0}^{\ell}x^2\dd x \\
		                  &= \uu{\frac{\ell^3}{3M}}
	\end{align*}
\end{solution}
