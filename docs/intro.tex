\section{Motivation}
Exercises are saved as separate files containing part problems. These files can be used to make sets, and you can cherry-pick or exclude certain part problems as you see fit. This makes it easier to maintain and keep your exercises flexible as the syllabus changes.

\section{Flow/Moderate start}
I suspect that working with this package will break you current flow. So let's go throught it.

This package assumes you put all of your exercises within the folder named \texttt{exercises} (you can change the default folder using \refCom{setExercisesDir})
\begin{dispListing*}{title=exercises/myexercise.tex}
\begin{intro}
  This introduces our problem
\end{intro}
\begin{problem}
  This is a partproblem 1,
  and will be hidden (just wait, you'll see)
\end{problem}
\begin{problem}
  This is a partproblem 2.
  This will not be hidden, but become part problem a!
\end{problem}
\end{dispListing*}
Let's build all of them first. In the main file, (the one where you include this package):
\begin{dispListing*}{title=main.tex}
  \documentclass{article}
  \usepackage{exercisebank}
  \makeset{myExerciseSet}{myexercise}
  \begin{document}
    \buildset{myExerciseSet}
  \end{document}
\end{dispListing*}
This builds the entire set, and adds Problem header and partproblem counters ( (1a) and (1b) ) by default.
\subsection{Select}
Now, let's build only the second problem.
\begin{dispListing*}{title=main.tex}
  \documentclass{article}
  \usepackage{exercisebank}
  \makeset{myExerciseSet}{\select{myexercise}{2}}
  \begin{document}
    \buildset{myExerciseSet}
  \end{document}
\end{dispListing*}
This should only build the intro and the one exercise you \refCom*{select}ed!

Now, say you want to hide the intro. Well all you have to do in this case is
make the package treat the intro as a problem in regards to what is \refCom*{select}ed.
Just add the optional argument \oarg{intro} to \dac{make}. That is switch
\begin{dispListing}
\makeset{myExerciseSet}{\select{myexercise}{2}}
\end{dispListing}
with
\begin{dispListing}
\makeset[intro]{myExerciseSet}{\select{myexercise}{3}}
\end{dispListing}
Notice that there are 3 `partproblems' now since we have to count the intro!

\subsection{Exclude}
But what if you have an exercise with 12 partproblems, and you only want to exclude the 7th partproblem? Well, then \dac{Exclude} is here to rescue the day for you.
\begin{dispListing}
  \makeset{myExerciseSet}{\exclude{soManyExercises}{7}}
\end{dispListing}
Here it's important to note that the [intro] argument would not make the intros disappear. If we wanted to only exclude the intro from our previous example file \texttt{exercises/myexercise.tex} we would do
\begin{dispListing}
\makeset[intro]{myExerciseSet}{\exclude{myexercise}{1}}
\end{dispListing}
So we're excluding the partproblem 1. But that's the intro when we send the [intro] optional argument
\subsection{Multiple}
In \dac{makeset} you can just separate exercises with commas! Here is an example:

Let's say you have two files with exercises. One located in \texttt{exercises/circuits/RLC.tex} and one in \texttt{exercises/ohm/ohmsGeneralLaw.tex}, and you want to include partproblem 1 through 5 from \texttt{RLC.tex} and all of the exercises from \texttt{ohmsGeneralLaw.tex}.

\begin{dispListing}
\makeset{\select{circuits/RLC}{1,...,5}, ohmsGeneralLaw}
\end{dispListing}
This will divide it up with problem headers. So that what is in the \texttt{RLC.tex}-file will be Problem 1, and \texttt{ohmsGeneralLaw.tex} Problem 2.
\subsection{Mixnmatch}
What if you want to make both of them the same exercise? Well, then you pass the [nohead] argument to \dac{makeset}:
\begin{dispListing}
\makeset[nohead]{\phead, \select{circuits/RLC}{1,...,5}, ohmsGeneralLaw}
\end{dispListing}
The \dac{phead} command makes a problem header. You can pass them as much as you want:

\begin{dispListing}
\makeset[nohead]{\phead, \select{circuits/RLC}{1,...,5},
              ohmsGeneralLaw, \phead, someOtherExercise, moreExercises}
\end{dispListing}

\subsection{Solutions}
The last thing to cover then is solutions. In your exercise files you just use the solution environment
\begin{dispListing}
\begin{solution}
Solution goes here
\end{solution}
\end{dispListing}
They are hidden by default, so you would have to use \dac{DisplaySolutions} in
your main file to display them.

That covers the basics. Enjoy
\begin{marker}
\dac{begin}\{problem\},\dac{end}\{problem\},\\
\dac{begin}\{solution\},\dac{end}\{solution\},\\
\dac{begin}\{intro\} and \dac{end}\{intro\} has to be on their own line without any spaces!
\end{marker}
